		The Data Trasport Network is the core part of the SCAD architecture. The 32 Fuctional-Units are identified by its physical address in the architecture which ranges from 0 to 31. Each Functional-Unit has output 
		that can send the data packets to any other Functional-Unit in a given point of time. A Data Packet structure is shown in the Fig.3. The intention is to get all the data packets deliverd to the 
		corresponding target Functional-Units in a minimum time, possibly constant time. A Cross-Bar switch network is a possible solution with a gurantee of constant time delivery of Data Packets to the target,
		but we have to compramise the resource consumption which makes it fairly complicated for 32 Functional Units. In other words the 32 X 32 cross connection itself will consume a major part of the resources, 
		which is not efficient. Another possible solution for this is to use memory mapped bus. But in this case, again it does not guarantee a constant time delivery to the 
		Functional-Units , since there will be cases where we have to prioritize the Data-Packets based in addresses possibly by using an Arbeiter. So as a tradeoff between resource consumption and time to deliver
		a Bitonic Network and a Benes Network are a good choices among many parallel sorting network. We implemented the DTN with a Bitonic as well as with a Benes Network. 
		A Bitonic Network sort any possible permutations in of input in constant time which makes it an excellent choice as a Network Router.
		Basically all the Fuctional-Units can be connected to each outputs of a Bitonic sorter with respect to the physical address and thus Bitonic Sorter acts Network Router. Before briefing further on the actual implementation
		we would like to give a short introduction about Bitonic-Network.

			\begin{figure}[!ht]
				%\begin{center}
				%    \begin{tabu} to 0.8\textwidth { | X[4] | X[6] |  X[4] | X[6] | X[4] | X[2] | }
				%    \hline
				%    \texttt{Valid bit \newline (1)} & \texttt{Target address\newline(5)} & \texttt{Source Address\newline(5)} & \texttt{Target buffer\newline(1)} & \texttt{Source Buffer\newline(1)} & \texttt{Data\newline(32)} \\
				%    \hline
				%    \end{tabu}
				%  \end{center}
				\includegraphics[width=\linewidth]{Data_Packet.png}
				\caption{Data Packet struture}
			\end{figure}


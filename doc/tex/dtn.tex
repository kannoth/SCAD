		
		
		The Data Trasport Network is the core part of the SCAD architecture. The 32 \textit{Fuction-Units} are identified by its physical address in the architecture which ranges from 0 to 31. Each \textit{Function-Unit} has output 
		that can send the \textit{Data-packets} to any other \textit{Function-Unit} in a given point of time. The \textit{Data-Packet} structure is shown in the Figure \ref{fig:Data_Packet}. The intention is to get all the \textit{Data-packets}
		deliverd to the corresponding target \textit{Function-Units} in minimum time, possibly constant time. A constant time means the time taken is deterministic. It can be several clock cycles and it will be the same for 
		any permutations of set if input addresses.
		
		A \textit{Cross-Bar} switch network is a possible solution with a gurantee of constant time delivery of \textit{Data Packets} to the target, but we have to compramise the resource consumption which makes it fairly complicated 
		for 32 \textit{Function-Units}. In other words the $32 X 32$ cross connection itself will consume a major part of the resources of the complete system, which is not efficient. Another possible solution for this is to use 
		memory mapped bus. But in this case, again it does not guarantee a constant time delivery to the \textit{Function-Units}, since there will be cases where we have to prioritize the \textit{Data-Packets} based in addresses possibly 
		by using an Arbeiter. 
		
		As a tradeoff between resource consumption and time to deliver a \textit{Data-Packet}, a {Bitonic Network} and a {Beneš Network} are indentified as good choices among many parallel sorting networks. 
		We implemented the DTN with a Bitonic as well as with a Beneš Network. A {Bitonic Network} sort any possible permutations in of input in constant time which makes it an excellent choice as a network router.
		Basically all the \textit{Fuction-Units} can be connected to each outputs of a \textit{Bitonic sorter} with respect to the physical address and thus {Bitonic sorter} acts network router. 
		Before going deep on the actual implementation we would like to give a short introduction about {Bitonic sorter}.

			\begin{figure}[!ht]
				%\begin{center}
				%    \begin{tabu} to 0.8\textwidth { | X[4] | X[6] |  X[4] | X[6] | X[4] | X[2] | }
				%    \hline
				%    \texttt{Valid bit \newline (1)} & \texttt{Target address\newline(5)} & \texttt{Source Address\newline(5)} & \texttt{Target buffer\newline(1)} & \texttt{Source Buffer\newline(1)} & \texttt{Data\newline(32)} \\
				%    \hline
				%    \end{tabu}
				%  \end{center}
				\includegraphics[width=\linewidth]{Data_Packet.png}
				\caption{Data Packet struture}
			\label{fig:Data_Packet}
			\end{figure}


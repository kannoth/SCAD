	\section{Instruction Set Architecture}
		The basic instruction set architecture of a SCAD machine is similar to that of a transport triggered architecture CPU.
		
		Two possible means of getting initial data to the functional units were considered in design.
		The first, having dedicated input and output FIFOs, was discarded in favor of an instruction set extension inspired by the bachelor thesis of Sebastian Schumb\cite{Schu15}.
		
		Aside from the mandatory move instruction, our Instruction Set Architecture (ISA) needs to have instructions for loading immediate values, jumping to fixed addresses and branching.
		An overview of those instructions is given in Figure \ref{fig:instruction_table}.
		
		\begin{figure}[!ht]
			\begin{center}
				\begin{tabular}{| l | p{8cm} |}
					\hline
					\textbf{instruction} & \textbf{semantics} \\ \hline
					\lstinline{move src, dest} & Move data from an output buffer to an input buffer by sending this instruction to the corresponding functional units. Data move will asynchronous \\ \hline
					\lstinline{jump address} & Set PC to address. \\ \hline
					\lstinline{immediate data} & Place data into output buffer of control unit. \\ \hline
					\lstinline{branch address} & A no-op when the first data in the input buffer is a 0, jump to address otherwise. Will wait for data to arrive when there is none. \\ \hline
				\end{tabular}
				\label{fig:instruction_table}
				\caption{SCAD Architecture Instructions}
			\end{center}
		\end{figure}


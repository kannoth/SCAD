
		A simple synchronous Control Unit(CU) is implemented to provide synchronous control.
The role of the control unit is to fetch instructions from the instruction memory, and broadcast it to the Functional Units via the Move Instruction Bus (MIB). For this purpose the CU contains Instruction Fetch (IF), Program Counter(PC) and  CU to MIB blocks.

\begin{figure}[!h]

\includegraphics[scale=0.5]{CU_Sys.png}

\end{figure}


		In classical sense, the program counter maintains the counter for the instruction address.
In case of branch instructions, this counter receives a branch offset which is for relative addressing. Table \ref{table:pc_description} describes the input/output ports.		

	\begin{table}[!h]
		\begin{tabular}{| c| c | c | p{9cm} |}
			\hline
			\textbf{name} & \textbf{direction} & \textbf{type} &  \textbf{description}\\ \hline			
			active & input & STD\_LOGIC & Signal from Ctrl\_Mov\_Instr. The signal is used to stall			 program counter increment in case of stall.  \\ \hline
			taken & output & STD\_LOGIC & Signal shows whether branch has been taken or not.  \\ \hline
			branch\_offset & input & data\_word & Branch offset for pc if jump is taken.  \\ \hline			
			pc & output & data\_word & Instruction address.  \\ \hline
			
				
		\end{tabular}
		
		\caption{PC ports \label{table:pc_description}}
		\centering
	\end{table}

		As per the name, the instruction fetch block fetches instruction from an asynchronous instruction memory.The instructions are then sent to Ctrl\textunderscore Mov\textunderscore Instr. In case of a stall signal from Functional Units, the Instruction fetch does not fetch any new instructions or issue new instruction to the Ctrl\textunderscore Mov\textunderscore Instr.Table \ref{table:instrfetch} describes the input/output ports.
\begin{table}[!h]
	\begin{tabular}{| c| c| c| p{9cm} |}
	\hline
	\textbf{name} & \textbf{direction} & \textbf{type} &  \textbf{description}\\ \hline			
	read\_en & output &  STD\_LOGIC & Enable Signal for memory.\\  \hline
	mem\_addr & output &  data\_word & Address to Read from Memory.\\  \hline
	pc\_addr & input &  data\_word & Instruction Address from PC.\\  \hline		
	stall &  input &  STD\_LOGIC & Stall Signal.\\  \hline
	\end{tabular}
	\caption{Instruction Fetch ports \label{table:instrfetch}}
	\centering
\end{table}


		Ctrl\textunderscore Mov\textunderscore Instr block is responsible for broadcasting instructions via the Move Instruction Bus. Ctrl\_Mov\_Instr uses a 2 step commit mechanism to account for stalls generated by the functional units. 	Ctrl\_Mov\_MIB ensures the Control Unit is stalled if the source or target of the current instruction generate a stall. The Control Unit will be stalled until these stall signals do not drop. Table \ref{table:ctrlToMib} describes the input/output ports.
\begin{table}[!h]
	\begin{tabular}{| c| c| c| p{9cm} |}
	\hline
		\textbf{name} & \textbf{direction} & \textbf{type} &  \textbf{description.}\\ \hline		
		instruction\_input & input & instruction & Instruction word from Instruction Fetch.\\ \hline
		ctrl\_mib & output & mib\_ctrl\_out & Data packet from Control Unit to MIB.\\ \hline
		active  & output & std\_logic & Enable signal, connected to stalls of Control Unit components.\\ \hline
		valid\_IF & input & STD\_LOGIC & Signals new instruction from IF.\\ \hline
		dtn\_data\_in	 & input & data\_port\_sending & Branch conditions are redirected to Control Unit via this signal.\\ \hline
		dtn\_data\_out & output & data\_port\_sending & Immediate arguments are redirected to Control Unit via this signal.\\ \hline
		stall & input & mib\_status\_bus & signal description.\\ \hline
	\end{tabular}
	\caption{Instruction Fetch ports \label{table:ctrlToMib}}
	\centering		
\end{table}		



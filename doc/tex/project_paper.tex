\documentclass[adraft]{eptcs}
\usepackage[utf8]{inputenc}

\usepackage[toc,page]{appendix}

% Defines \FloatBarrier
% The "section" parameter makes floats stay in the section they are defined in.
\usepackage[section]{placeins}

\usepackage{graphicx}
%\usepackage{listings}
\usepackage{eslistings}
\usepackage{parcolumns}
\usepackage{amsmath}
\usepackage{url}
\usepackage{breakurl}
\usepackage{todonotes}

\usepackage{tikz-timing}

\renewcommand{\abstractname}{Abstract}

% Define Language
\lstdefinelanguage{SCAD}
{
  % list of keywords
  morekeywords={
    immediate,
    move,
    branch,
    jump
  },
  sensitive=false, % keywords are not case-sensitive
  morecomment=[l]{//}, % l is for line comment
  morecomment=[s]{/*}{*/}, % s is for start and end delimiter
  morestring=[b]" % defines that strings are enclosed in double quotes
}

%TODO: center listings
\lstset{language=SCAD,frame=L,style=ESStyle,tabsize=2,basicstyle=\small\ttfamily}
%\lstset{language=C,frame=L,style=ESStyle,tabsize=2,basicstyle=\small\ttfamily}
%\lstset{language=C,frame=L,basicstyle=\small\ttfamily,tabsize=2}
\lstset{numbersep=10pt}
\lstset{xleftmargin=5.0ex}
\lstset{numbers=left}
% Captions to the bottom
\lstset{captionpos=b}
\lstset{breaklines=false}

% Not used and has a tendency to break code.
\lstset{mathescape=false}

\newcommand{\ie}{i.~e.}
\newcommand{\Ie}{I.~e.}
\newcommand{\eg}{e.g.}
\newcommand{\Eg}{E.~g.}



\title{SCAD Architecture Project Paper}

\author{Julius Roob, ...
	\institute{University of Kaiserslautern, Embedded Systems Group}
	\email{julius@juliusroob.de}, ...
}

\begin{document}
	\maketitle \newpage
	\tableofcontents \newpage
	\listoftodos \newpage
	
		\section{Instruction Set Architecture}
		Like the Transport Triggered Architecture (TTA), the main feature of the SCAD machine instruction set is the move instruction.
		Moves happen from the output buffer of one Functional Unit (FU) to the input of another.
		Those moves have an order given by the program, and all parallel or out-of-order execution is done transparently by the hardware.
		
		While all operations are performed by moving data between functional units, some initial data is required, for example the addresses of where inputs and results are located in memory.
		Two possible means of getting that initial data to the functional units were considered in design.
		Those two were to either have dedicated FIFOs for inputs and outputs of the processor and program, or extend the ISA by instructions to load values for the data network.
		For the first approach, all constants that are required for a program to run have to be made available through one or more FIFOs or ROMs that outputs data into the data network when a corresponding move instruction is issued.
		The second, which is inspired by the bachelor thesis of Sebastian Schumb \cite{Schu15}, is to add at least one instruction to load immediate values.
		
		To make this design as simple to implement as possible, it was decided to have the control unit be part of the data network like the FUs, and both load immediate values into an output buffer, and take branch conditions from and input buffer. %This is explained further in Section \ref{?}\todo{Add Section reference}.
		
		So, aside from the mandatory move instruction, the ISA has instructions for loading immediate values, jumping to fixed addresses and branching.
		An overview of those instructions is given in Figure [1].%\ref{fig:instruction_table}.
		
		Example programs can be found in the Appendices \ref{app:simple} and \ref{app:fib}.
		
		
		\begin{figure}[!ht]
			\begin{center}
				\begin{tabular}{| l | p{8cm} |}
					\hline
					\textbf{instruction} & \textbf{semantics} \\ \hline
					move src, dest & Move data from an output buffer to an input buffer by sending this instruction to the corresponding functional units. Data move will asynchronous \\ \hline
					jump address & Set PC to address. \\ \hline
					immediate data & Place data into output buffer of control unit. \\ \hline
					branch address & A no-op when the first data in the input buffer is a 0, jump to address otherwise. Will wait for data to arrive when there is none. \\ \hline
				\end{tabular}
				\label{fig:instruction_table}
				\caption{SCAD Architecture Instructions}
			\end{center}
		\end{figure}


	
	\section{Move Instruction Bus}
		\subsection{2-Phase Commit}
			To take into account both stalls of source and destination functional units, the control unit sends move instructions in two phases, both of which are indicated by a rising edge of the "valid" flag\todo{Either the text, the diagram and/or the implementation need to be adapted (not in sync).}.
			First, with phase low, the functional units only check whether there is space in the corresponding input and output buffers.
			When stalls are asserted, the control unit waits some time until retrying.
			When no functional unit stalls, the phase being high signals a "write".
			
			\begin{figure}
				\begin{center}
					\begin{tikztimingtable}
						src & Z18D{source address}Z \\
						dest & Z18D{destination address}Z \\
						phase & LLLLLLLLLLLLLLHHHHLL \\
						valid & LLLHHHHHHHHHHHHHHHLL \\
						src\_stalled & LLLLLLLLLLLLLLLLLLLL \\
						dest\_stalled & LLLLLHHHHHLLLLLLLLLL \\
					\end{tikztimingtable}
					\caption{2-Phase Commit on Move Instruction Bus with Destination Stalling}
				\end{center}
			\end{figure}

	\section{Control Unit and Data Network}
		To keep the MIB simple, we will have the control unit send immediate values, and receive branch conditions through the data network.
		While broadcasting immediate values through either the MIB or a separate bus may be the faster alternative, having the control unit send them through the data network keeps the architecture cleaner\todo{better formulation please}.



	\section{Future Work}
		Having duplication as a seperate functional unit may cause it to be a significant bottleneck.
		One feasible solution to this is the extension of the move instruction by an additional "non-destructive" move, where data sent is kept in the output buffer of the sender.
		
		\todo[inline]{at least two more ideas}

	\newpage
	\section{Bibliography}
		%\nocite{*}
		\bibliographystyle{eptcs}
		\bibliography{references}
	\newpage
	
	\begin{appendices}
		\newpage
		\section{Memory Access and Branch}
			\label{app:simple}
			Basic example for memory access and branching:
			\lstinputlisting{assembler/simple.asm}
		
		\newpage
		\section{Fibonacci}
			\label{app:fib}
			\lstinputlisting{assembler/fibonacci.asm}
	\end{appendices}
	
\end{document}


\begin{abstract}
Current processor architectures offer instruction-level parallelism by duplicating processing resources and handling both instruction and data distribution from a central controller. Although this scheme offers high throughput on non-dependent instructions, existence of central register file cripples the performance on data dependent workloads where several consecutive instructions contend registers, which results in serialization of the machine code and stalls on controller. 

In this work, we implement \emph{Synchronous-data asynchronous dataflow (SCAD)} architecture, where instructions are issued in order but data movement is handled by processing elements that are connected to each other via data network. Similar to out-of-order superscalar processors, SCAD architecture enforces dataflow order of instructions. However, fundamental difference is the de-centralized hazard resolving mechanism, in which the controller does not halt issuing instructions as long as there are enough processing elements available. Instructions are issued in-order (synchronously), while operands and results are moved out-of-order (asynchronously). Data dependencies within a block of operations are resolved with a VLIW-like mechanism without incurring register file contention. 
\end{abstract}
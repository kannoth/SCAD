

\begin{frame}
	\frametitle{Functional Units}
	\itemize{
		\item Functional units (FU) are the fundamental processing elements in SCAD architecture
		\item Each FU stores its input operands and output result in buffers
		\item New operations can be added into processor with minimal amount of micro-architectural design effort
		\item In current stepping level, there are two classes of FUs : ALU and memory
	}
\end{frame}


\begin{frame}
\frametitle{Functional Units : ALU}
	\includegraphics[width=\linewidth,height=0.9\textheight,keepaspectratio]{alu_fu.png}
\end{frame}

\begin{frame}
\frametitle{Functional Units : ALU}
	\itemize{
			\item ALU FUs include two buffers for operands and one FIFO buffer for result
			\item Operation to be performed can easily be realized by implementing the well-defined
			ALU operation interface
			\item Single-cycle, multi-cycle or pipelined operations can be added into 
			processor, resulting in easy expansion of ISA !
		}
\end{frame}

\begin{frame}
\frametitle{Functional Units : Memory}
\begin{columns}
\begin{column}{0.48\textwidth}
	\begin{figure}[htbp]
		\centering
		\def\svgscale{0.40}
		\input{figures/load_fu.pdf_tex}
		\caption{Load FU}
		\label{fig:load_fu} 
	\end{figure}
\end{column}
\begin{column}{0.60\textwidth}
	\begin{figure}[htbp]
		\centering
		\def\svgscale{0.28}
		\input{figures/store_fu.pdf_tex}
		\caption{Store FU}
		\label{fig:store_fu} 
	\end{figure}
\end{column}
\end{columns}
\end{frame}

\begin{frame}
		\frametitle{Functional Units : Memory}
\begin{columns}
	\begin{column}{0.48\textwidth}
		\itemize{
			\item Memory is divided into banks, where each bank is accessed by a 
			load/store FU pair
			\item Bank controller serializes accesses to banks by competing load/store
			operations : writes are given precedence
			\item Due to asynchronous data movement, enforcing even a slight consistency
			model seems hard (not formally, but implementation-wise)
		}
	\end{column}
	\begin{column}{0.60\textwidth}
		\begin{figure}[htbp]
			\centering
			\def\svgscale{0.20}
			\input{figures/memory_top.pdf_tex}
			\caption{Top-level memory architecture}
			\label{fig:mem_top} 
		\end{figure}
	\end{column}
\end{columns}	
	


\end{frame}